% $test: true
% $date: 2016-01-22
% $test$duration: [60, минут]

\section*{Экзамен I. Геометрия}

% $test$mark-limits: {3: 4, 4: 7, 5: 10}

\subsection*{Ещё вариант}

\begin{problems}

% >> $test$problem-scores: [[2, 2]]
\item
В треугольнике $ABC$ проведены чевианы $A A_1$, $B B_1$, $C C_1$,
пересекающиеся в~точке~$P$.
Прямые $B_1 C_1$ и $A A_1$ пересекаются в~точке~$N$.
Известно, что $A B_1 = B_1 C$ и $B_1 N = N C_1$.
\\
\subproblem
Пусть $S_{B_1 P C_1} = 1$.
Найдите $S_{A P C_1}$, $S_{A P B_1}$, $S_{C P B_1}$.
\\
\subproblem
Докажите, что $P$~--- точка пересечения медиан треугольника $ABC$.

% >> $test$problem-scores: [[2, 2]]
\item
Одна окружность вписана в прямоугольную трапецию, а вторая касается большей
боковой стороны и продолжений оснований.
\\
\subproblem
Докажите, что расстояние между центрами окружностей равно большей боковой
стороне трапеции.
\\
\subproblem
Найдите расстояние от вершины одного из прямых углов трапеции до центра второй
окружности, если точка касания первой окружности с большей боковой стороной
трапеции делит ее на отрезки, равные $2$ и $50$.

% >> $test$problem-scores: [[2, 2]]
\item
Две окружности пересекаются в~точках $P$ и~$Q$.
Прямая, проходящая через точку~$P$, второй раз пересекает первую окружность
в~точке~$A$, а~вторую~--- в~точке~$D$.
Прямая, проходящая через точку~$Q$ параллельно $AD$, второй раз пересекает
первую окружность в~точке~$B$, а~вторую~--- в~точке~$C$.
\\
\subproblem
Докажите, что четырехугольник $ABCD$~---параллелограмм.
\\
\subproblem
Найдите отношение $BP : PC$, если радиус первой окружности вдвое больше радиуса
второй.

\end{problems}

