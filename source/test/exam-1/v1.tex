% $test: true
% $date: 2015-12-23
%% $test$duration: [240, минут]

\section*{Экзамен I}

\emph{Экзамен состоит из~трех частей, за~каждую из~которых выставляется
отдельная оценка. Общая длительность экзамена~--- 240 минут.}

\subsection*{Алгебра}

\begin{problems}

\itemy{a1}
Дан многочлен\enspace
$(1 - 2 x^2 + 3 x^3)^7$.
Найдите
\\
\subproblem
сумму коэффициентов при четных степенях $x$;
\\
\subproblem
сумму коэффициентов при нечетных степенях $x$;
\\
\subproblemx{*}
коэффициент при $x^{11}$.

\itemy{a2}
Решите уравнение\enspace
$\sqrt{x + 1} - \sqrt{2 x - 5} = \sqrt{x - 2}$.

\itemy{a3}
Вычислите\enspace
\(
%    \bigl(
        \sin^4(\alpha) - \cos^4(\alpha)
%    \bigr)
\),\enspace
если $\tg(\alpha / 2) = 1 / 2$.

\itemy{a4}
Решите уравнение\enspace
$\cos(12 x) - 2 \sin^2(3 x) - 1 = 0$.

\end{problems}


\subsection*{Геометрия}

\begin{problems}

\itemy{g1}
В~треугольнике $ABC$ проведены чевианы $A A_1$, $B B_1$, $C C_1$,
пересекающиеся в~точке~$P$.
Известно, что $A B_1 = B_1 C$, $A C_1 : C_1 B = 1 : 2$.
\\
\subproblem
Докажите, что $A_1 C_1$ параллельно $AC$.
\\
\subproblem
Пусть $T$~--- точка пересечения $A_1 C_1$ и~$B B_1$.
Найдите $A_1 T : T C_1$ и~$B T : T B_1$.

\itemy{g2}
К~двум непересекающимся окружностям равных радиусов проведены две параллельные
касательные.
Окружности касаются одной из~этих прямых в~точках $A$ и~$B$.
Через точку $C$, лежащую на~отрезке $AB$, проведены касательные к~этим
окружностям, пересекающие вторую прямую в~точках $D$ и~$E$, причем отрезки $CA$
и~$CD$ касаются одной окружности, а~отрезки $CB$ и~$CE$~--- другой.
\\
\subproblem
Докажите, что периметр треугольника $CDE$ вдвое больше расстояния между
центрами окружностей.
\\
\subproblem
Найдите $DE$, если радиусы окружностей равны $5$, расстояние между их центрами
равно $18$, а~$AC = 8$.

\itemy{g3}
Диагонали $AC$ и~$BD$ четырехугольника $ABCD$, вписанного в~окружность,
пересекаются в~точке $P$, причем $BC = CD$.
\\
\subproblem
Докажите, что $AB : BC = AP : PD$.
\\
\subproblem
Найдите площадь треугольника $COD$, где $O$~--- центр окружности, вписанной
в~треугольник $ABD$, если дополнительно известно, что $BD$~--- диаметр
описанной около четырехугольника $ABCD$ окружности, $AB = 6$, а~$BC = 6
\sqrt{2}$.

\end{problems}

\begin{flushright}\scriptsize
\((1+2)+(1+2)+(1+2) = 9\)%
\\
\(3 \mapsto \mathbf{3},\ 5 \mapsto \mathbf{4},\ 7 \mapsto \mathbf{5}\)%
\end{flushright}

\subsection*{Математический анализ}

\begin{problems}

\itemy{c1}
Найдите все пары действительных $a$ и~$b$, при которых справедливо равенство
\[
    \lim\limits_{n \to \infty}
        \frac{(2 a - b) n^2 + (a + 3 b) n + 5}{a^2 n + a - 1}
=
    -2
\, . \]

\itemy{c2}
Приведите пример двух последовательностей $x_n$ и~$y_n$, не~имеющих конечных
пределов, таких, что и~их сумма $a_n=x_n+y_n$, и~их произведение $b_n=x_ny_n$
имеют конечный предел.

\itemy{c3}
Сформулируйте и~докажите теорему Вейерштрасса об~ограниченной возрастающей
последовательности.

\end{problems}

