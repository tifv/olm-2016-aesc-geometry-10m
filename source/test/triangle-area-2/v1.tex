% $test: true
% $date: 2015-10-02
% $test$duration: [45, минут]

\section*{Контрольная работа: Метод площадей --- 2}

% $test$mark-limits: {3: 6, 4: 8, 5: 10}

% $matter[-contained]:
% - verbatim: \setcounter{figure}{0}
% - .[contained]


\begin{problems}

% >> $test$problem-scores: [3]
\item
В треугольнике $ABC$ биссектриса~$AL$ делит сторону~$BC$ в отношении $2 : 1$.
В каком отношении медиана~$CM$ делит эту биссектрису?

% >> $test$problem-scores: [[1, 1, 2]]
\item
В треугольнике $ABC$ чевианы $A A_1$, $B B_1$, $C C_1$ пересекаются
в~точке~$P$ (внутри треугольника).
Даны отношения $AP : P A_1 = 3 : 2$ и $BP : P B_1 = 4 : 1$.
Найдите отношения
\\
\subproblem $A B_1 : B_1 C$;
\quad
\subproblem $CP : P C_1$;
\\
\subproblem $CP : PT$, где $T$~--- точка пересечения $C C_1$ и $A_1 B_1$.

%\item
%\label{test/triangle-area-2:problem:bisectors}%
%Биссектрисы $AL$ и $BK$ треугольника $ABC$ пересекаются в~точке~$I$
%(рис.~\ref{test/triangle-area-2:problem:bisetors:fig}).
%Докажите справедливость формулы
%\(
%    AI : IL = (b + c) : a
%\),
%где $a$, $b$, $c$~--- соответственные стороны треугольника.

%\begin{figure}[hb]
%\leavevmode\null\hfill
%    \begin{subfigure}{0.4\textwidth}
%    \begin{center}
%        \%jeolmfigure[height=0.75\linewidth]{bisectors}
%        \caption{к задаче~\ref{test/triangle-area-2:problem:bisectors}.}
%        \label{test/triangle-area-2:problem:bisectors:fig}
%    \end{center}
%    \end{subfigure}
%\hfill
%    \begin{subfigure}{0.4\textwidth}\centering
%    \begin{center}
%        \%jeolmfigure[height=0.75\linewidth]{exscribed}
%        \caption{к задаче~\ref{test/triangle-area-2:problem:exscribed}.}
%        \label{test/triangle-area-2:problem:exscribed:fig}
%    \end{center}
%    \end{subfigure}
%\hfill\null\par
%    \caption{}
%\end{figure}

% >> $test$problem-scores: [4]
\item
\label{test/triangle-area-2:problem:exscribed}%
Вневписанная окружность треугольника $ABC$ касается стороны $BC$ и продолжений
$AB$ и $AC$ в точках $T$, $N$, $K$ соответственно.
Точка $Q$~--- точка пересечения $BK$ и $CN$.
%(рис.~\ref{test/triangle-area-2:problem:exscribed:fig}).
Докажите, что точки $A$, $T$ и $Q$ принадлежат одной прямой.

% >> $test$problem-scores: [4]
\item
\label{test/triangle-area-2:problem:square}%
Рис.~\ref{test/triangle-area-2:problem:square:fig}.
Найдите отношение закрашенной площади к~площади всего квадрата.

\begin{figure}[hp]
\begin{center}
    \jeolmfigure[width=0.25\linewidth]{square}
    \caption{к задаче~\ref{test/triangle-area-2:problem:square}.}
    \label{test/triangle-area-2:problem:square:fig}
\end{center}
\end{figure}

%\item
%\label{test/triangle-area-2:problem:modified-van-obel}%
%Рис.~\ref{test/triangle-area-2:problem:modified-van-obel:fig}.
%Докажите теорему:\enspace
%\(
%    PS : SL = PR : RN + PQ : QK
%\).
%
%\begin{figure}[hb]
%\begin{center}
%    \%jeolmfigure[width=0.3\linewidth]{van-obel-modified}
%    \caption{к задаче~\ref{test/triangle-area-2:problem:modified-van-obel}.}
%    \label{test/triangle-area-2:problem:modified-van-obel:fig}
%\end{center}
%\end{figure}

\end{problems}

