% $date: 2016-04-22
% $test: true
% $test$duration: [20, минут]

\section*{Добрая контрольная работа: Плоскости и прямые --- 1}

% $test$mark-limits: {3: 2, 4: 4, 5: 5}

\emph{%
Использовать планиметрические утверждения можно, только указав плоскость,
в~которой они работают.}

\begin{problems}

% >> $test$problem-scores: [2]
\item
В~пространстве дан равносторонний треугольник $ABC$, точка~$M$~--- его центр.
Плоскость~$\alpha$ проходит через $B$ и~пересекает прямую~$AC$ в~точке~$R$.
Чему равно $AR : RC$, если $AM \parallel \alpha$?

%% >> $test$problem-scores: [3]
%\item
%Точка~$C$ лежит на~отрезке~$AB$.
%Через точку~$A$ проведена плоскость, а~через точки $B$ и~$C$~--- параллельные
%прямые, пересекающие эту плоскость соответственно в~точках $B_1$ и~$C_1$.
%Найдите длину отрезка $C C_1$, если $AC : CB = 3 : 2$ и~$B B_1 = 20$.

% >> $test$problem-scores: [[2, 2]]
\item
\subproblem
Три прямые попарно пересекаются.
Докажите, что они либо все проходят через некоторую точку, либо все лежат
в~некоторой плоскости.
\\
\subproblem
\ldots по~индукции докажите то~же для $n$ попарно пересекающихся прямых
($n \geq 3$).

\end{problems}

