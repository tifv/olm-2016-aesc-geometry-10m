% $test: true
% $date: 2015-11-13

\section*{Срез}

% $test$duration: [100, минут]

% $style:
% - ..
% - verbatim: \def\jeolmgroup{Математика 10М}

% $build$matter[print]: [[.], [.]]

\begingroup
    \ifx\mathup\undefined
        \def\piconst{\uppi}
    \else
        \def\piconst{\mathup{\pi}}

\subsection*{Алгебра}

\begin{problems}

\item
Докажите неравенство\enspace
$3^{n} - n > 2^{n} + n$\enspace
при $n > 1$.

\item
Упростите выражение
\\
\( \displaystyle
    \cos^6 \left( \alpha - \frac{  \piconst}{2} \right)
+
    \sin^6 \left( \alpha - \frac{3 \piconst}{2} \right)
-
    \frac{3}{4}
    \left(
        \sin^2 \left( \alpha + \frac{  \piconst}{2} \right)
        -
        \cos^2 \left( \alpha + \frac{3 \piconst}{2} \right)
    \right)^2
\).

\end{problems}

\subsection*{Геометрия}

\begin{problems}

\item
К окружности, вписанной в квадрат $ABCD$, проведена касательная, пересекающая
стороны $AB$ и $AD$ в точках $M$ и $N$ соответственно.
\\
\subproblem
Докажите, что периметр треугольника $AMN$ равен стороне квадрата.
\\
\subproblem
Прямая~$MN$ пересекает прямую~$CD$ в точке~$P$.
В~каком отношении делит сторону~$BC$ прямая, проходящая через точку~$P$
и центр окружности, если $AM : MB = 1 : 2$?

\item
В равнобедренном треугольнике $ABC$ с углом $120^\circ$ при вершине~$A$
проведена биссектриса~$BD$.
В треугольник $ABC$ вписан прямоугольник $DEFH$ так, что сторона~$FH$ лежит
на отрезке~$BC$, а вершина~$E$~--- на~отрезке~$AB$.
Докажите, что $FH = 2 DH$.

\end{problems}

\subsection*{Математический анализ}

\begin{problems}

\item
Постройте график функции\enspace
$y = 2 \sin(x + \piconst / 4) + 1$.

\item
Найдите\enspace
\( \displaystyle
    \lim_{n \to \infty}
        \frac{n^3 - n - 6}{-2 n^3 + n^2 - 17}
\).

\end{problems}

