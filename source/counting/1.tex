% $date: 2016-03-18

\section*{Счёт в геометрии}

% $authors:
% - Алексей Пономарёв

\claim{Обозначения}
В треугольнике $ABC$ соответственно
\begin{itemize}
    \item $a$, $b$, $c$~--- стороны $BC$, $CA$, $AB$;
    \item $m_{a}$, $m_{b}$, $m_{c}$~---
        длины медиан из вершин $A$, $B$, $C$;
    \item $h_{a}$, $h_{b}$, $h_{c}$~---
        длины высот из вершин $A$, $B$, $C$;
    \item $l_{a}$, $l_{b}$, $l_{c}$~---
        длины биссектрис из вершин $A$, $B$, $C$;
    \item $R$~--- радиус описанной окружности.
\end{itemize}

\subsection*{Теорема косинусов}

\begin{problems}

\item \emph{Вид треугольника.}
Укажите условия на $a$, $b$, $c$, которые соответствуют
\\
\subproblem остроугольному
\quad
\subproblem прямоугольному
\quad
\subproblem тупоугольному
\\
треугольнику.

\item
Известно, что $m_{a}^2 + m_{b}^2 = 5 m_{c}^2$.
Укажите вид треугольника.

\item
В треугольнике $ABC$ провели медианы $A A_1$ и $B B_1$, которые пересеклись
в точке~$M$.
Длины медиан равны соответственно $12$ и $9$, а угол $AMB$ равен $120^{\circ}$.
Найдите длину медианы $C C_1$.

\item
Оказалось, что $l_a = l_b$.
Докажите, что $a = b$.

\item
Оказалось, что
\(
    1 / b + 1 / c = 1 / l_a
\).
Найдите $\angle A$.

\item
В остроугольном треугольнике оказалось, что $h_a = l_b = m_c$.
Можно ли утверждать, что $a = b = c$?

\end{problems}

\subsection*{Теорема синусов}

\begin{problems}

\item
Существует ли треугольник, у которого
$\sin(\angle A) + \sin(\angle B) = \sin(\angle C)$?

\item
В треугольнике провели высоты $A A_1$ и $C C_1$.
Оказалось, что $A_1 C_1 = R / 2$.
Найдите все возможные значения $\angle B$.

\item
Две окружности радиусов $r_1$ и $r_2$ пересекаются в точках $A$ и $B$.
Прямая касается их в точках $C$ и $D$.
Найдите радиусы окружностей, описанных вокруг треугольниов $ACD$ и $BCD$.

\end{problems}

