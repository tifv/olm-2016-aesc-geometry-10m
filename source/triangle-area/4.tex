% $date: 2015-10-02

\section*{Метод площадей --- разнобой}

% $build$style[-print]:
% - .[colour-links]

% $matter[-preamble-package-guard]:
% - preamble package: subcaption
% - .[preamble-package-guard]

\begin{problems}

\item
\label{triangle-area-4:problem:two-triangles}%
Точки $T$, $N$ и~$D$ соответственно на~сторонах $BC$, $AC$ и~$AB$
треугольника $ABC$ таковы, что
\(
    AD : DB = BT : TC = CN : NA = k
\)
для некоторого $k < 1$.
Прямые $AT$, $BN$ и~$CD$ в~пересечении дают треугольник $QEF$
(рис.~\ref{triangle-area-4:problem:two-triangles:fig}).
Найдите отношение $S_{QEF}$ к~$S_{ABC}$.

\begin{figure}[hb]
\leavevmode\null\hfill
    \begin{subfigure}{0.4\textwidth}
    \begin{center}
        \jeolmfigure[height=0.75\linewidth]{two-triangles}
        \caption{к задаче~\ref{triangle-area-4:problem:two-triangles}.}
        \label{triangle-area-4:problem:two-triangles:fig}
    \end{center}
    \end{subfigure}
\hfill
    \begin{subfigure}{0.4\textwidth}\centering
    \begin{center}
        \jeolmfigure[height=0.75\linewidth]{trapezoid}
        \caption{к задаче~\ref{triangle-area-4:problem:trapezoid}.}
        \label{triangle-area-4:problem:trapezoid:fig}
    \end{center}
    \end{subfigure}
\hfill\null\par
    \caption{}
\end{figure}


\item
\label{triangle-area-4:problem:trapezoid}%
Точки $Q$ и~$T$~---соответственно середины $BC$ и~$AD$~--- оснований трапеции
$ABCD$ (рис.~\ref{triangle-area-4:problem:trapezoid:fig}).
На~продолжении диагонали~$AC$ отмечена точка~$K$.
Прямые $KQ$ и~$AB$ пересекаются в~точке~$E$, а~$KT$ и~$CD$~--- в~точке~$F$.
Докажите, что $EF \parallel AD$.

\item
\label{triangle-area-4:problem:modified-gergonne}%
\emph{(Теорема Жергонна из~альтернативной вселенной.)}
\\
Рис.~\ref{triangle-area-4:problem:modified-gergonne:fig}.
Докажите теорему:\enspace
\(
    - NL : KL + NR : PR + NQ : QS = 1
\).
%\\
%\emph{%
%Подсказка: как доказывалась~бы теорема Жергонна для треугольника $PSK$
%с~чевианами, пересекающимися в~точке~$N$?}

\begin{figure}[hb]
\begin{center}
    \jeolmfigure[width=0.3\linewidth]{gergonne-modified}
    \caption{к задаче~\ref{triangle-area-4:problem:modified-gergonne}.}
    \label{triangle-area-4:problem:modified-gergonne:fig}
\end{center}
\end{figure}

\item
Медиана~$BM$ и~биссектриса~$CL$ в~треугольнике $ABC$ пересекаются в~точке~$Q$.
Найдите отношение $CQ : QL$, если $\angle A = 30^\circ$, $\angle B = 45^\circ$.

\item
\emph{Напоминание.}
Пусть в~треугольнике даны длины сторон $BC = a$, $CA = b$, $AB = c$.
\\
\subproblem
\label{triangle-area-4:problem:gergonne-point}%
Найдите длины отрезков, на~которые стороны треугольника разбиваются точками
касания с~вписанной окружностью;
убедитесь в~существовании \emph{точки Жергонна}
(рис.~\ref{triangle-area-4:problem:gergonne-point:fig}).
\\
\subproblem
\label{triangle-area-4:problem:nagel-point}%
Найдите длины отрезков, на~которые стороны треугольника разбиваются точками
касания с~вневписанными окружностями;
убедитесь в~существовании \emph{точки Нагеля}
(рис.~\ref{triangle-area-4:problem:nagel-point:fig}).

\begin{figure}[hb]
\leavevmode\null\hfill
    \begin{subfigure}{0.4\textwidth}
    \begin{center}
        \jeolmfigure[height=0.75\linewidth]{gergonne-point}
        \caption{к задаче~\ref{triangle-area-4:problem:gergonne-point}.}
        \label{triangle-area-4:problem:gergonne-point:fig}
    \end{center}
    \end{subfigure}
\hfill
    \begin{subfigure}{0.4\textwidth}
    \begin{center}
        \jeolmfigure[height=0.75\linewidth]{nagel-point}
        \caption{к задаче~\ref{triangle-area-4:problem:nagel-point}.}
        \label{triangle-area-4:problem:nagel-point:fig}
    \end{center}
    \end{subfigure}
\hfill\null\par
    \caption{}
\end{figure}

\end{problems}

%\begin{figure}[hb]
%\begin{center}
%    \XXX jeolmfigure[width=0.3\linewidth]{../3/menelaus-0}
%    \caption{к задаче~\ref{triangle-area-4:problem:ceva-feast}.}
%    \label{triangle-area-4:problem:ceva-feast:fig}
%\end{center}
%\end{figure}

%\item
%\label{triangle-area-4:problem:ceva-feast}%
%Рис.~\ref{triangle-area-4:problem:ceva-feast:fig}.
%Дано $PK : KQ = 2 : 3$ и~$NL : LK = 5 : 6$.
%Решите каждый пункт однократным применением теоремы Чевы или ван Обеля
%(и~используя предыдущие пункты):
%\begingroup \parskip=0pt
%\begin{tabbing}
%\subproblem
%найдите $RN : NP$;
%\qquad\=
%\subproblem
%найдите $PS : SM$;
%\qquad\=
%\subproblem
%найдите $NS : SQ$;
%\\
%\subproblem
%найдите $KS : SR$;
%\qquad\>
%\subproblem
%найдите $PL : LS$.
%\end{tabbing}
%\endgroup % \parskip

