% $date: 2015-09-25

\section*{Теоремы ван Обеля и Жергонна}

% $build$style[-print]:
% - .[colour-links]

% $matter[-preamble-package-guard]:
% - preamble package: subcaption
% - .[preamble-package-guard]

\emph{Задачи со~звездочкой \textbf{*} менее обязательные, чем остальные.}


\subsection*{Теоремы Чевы и Менелая (повторение)}

\begin{figure}[hb]
\begin{center}
    \jeolmfigure[width=0.3\linewidth]{menelaus-0}
    \caption{к задаче~\ref{triangle-area-3:problem:1}.}
    \label{triangle-area-3:problem:1:fig}
\end{center}
\end{figure}

\begin{problems}

\item
\label{triangle-area-3:problem:1}%
Рис.~\ref{triangle-area-3:problem:1:fig}.
Решите каждый пункт однократным применением теоремы Чевы или Менелая
(и~используя предыдущие пункты):
\begin{tabbing}
\subproblem
зная $PK : KQ = 2 : 3$ и~$NS : SQ = 1 : 3$,
найдите $NL : LK$;
\\
\subproblem
\ldots найдите $RN : NP$;
\qquad\=
\subproblem
\ldots найдите $KS : SR$;
\\
\subproblem
\ldots найдите $PS : SM$;
\qquad\>
\subproblem
\ldots найдите $PL : LS$.
\end{tabbing}
\emph{Подсказка: во всех пунктах достаточно теоремы Менелая.}

\begin{figure}[hb]
\leavevmode\null\hfill
    \begin{subfigure}{0.42\textwidth}
        \jeolmfigure[height=0.5\linewidth]{menelaus-1}
        \caption{к~задаче~\ref{triangle-area-1:problem:2}.}
        \label{triangle-area-3:problem:2:fig}
    \end{subfigure}
\hfill
    \begin{subfigure}{0.3\textwidth}
        \jeolmfigure[height=0.7\linewidth]{menelaus-2}
        \caption{к~задаче~\ref{triangle-area-1:problem:3}.}
        \label{triangle-area-3:problem:3:fig}
    \end{subfigure}
\hfill\null\par
    \caption{}
\end{figure}

%\begin{figure}[hb]
%\begin{center}
%    \jeolmfigure[height=0.20\linewidth]{menelaus-1}
%    \caption{к~задаче~\ref{triangle-area-1:problem:2}.}
%    \label{triangle-area-3:problem:2:fig}
%\end{center}
%\end{figure}

\item
\subproblem
\label{triangle-area-1:problem:2}%
Докажите, что
на~рис.~\ref{triangle-area-3:problem:2:fig}
выполняется $A C_1 : C_1 B = A C_2 : C_2 B$.
\\
%\begin{figure}[hb]
%\begin{center}
%    \jeolmfigure[height=0.20\linewidth]{menelaus-2}
%    \caption{к~задаче~\ref{triangle-area-1:problem:3}.}
%    \label{triangle-area-3:problem:3:fig}
%\end{center}
%\end{figure}
\subproblemx{*}
\label{triangle-area-1:problem:3}%
Докажите, что точки $A_2$, $B_2$ и~$C_2$
на~рис.~\ref{triangle-area-3:problem:3:fig}
лежат на~одной прямой.

\end{problems}


\subsection*{Лемма о трёх параллелограммах}

\begin{problems}

\item
\label{triangle-area-3:problem:parallelogramm}%
\textit{Лемма о трёх параллелограммах.}
На~рис.~\ref{triangle-area-3:problem:parallelogramm:fig}
дан параллелограмм $ABCD$, прямые $a$ и~$b$ проходят через точку~$M$
параллельно его сторонам.
Точка $N$~--- пересечение прямых $BP$ и~$DQ$.
Докажите, что точки $C$, $M$ и~$N$ лежат на~одной прямой.

\begin{figure}[hb]
\begin{center}
    \jeolmfigure[width=0.3\linewidth]{parallelogram}
    \caption{к лемме о трех параллелограммах}
    \label{triangle-area-3:problem:parallelogramm:fig}
\end{center}
\end{figure}

\end{problems}


\subsection*{Теоремы ван Обеля и Жергонна}

\begin{figure}[hb]
\begin{center}
    \jeolmfigure[width=0.3\linewidth]{generic-triangle}
    \caption{%
        \emph{стандартные обозначения} в~этом разделе.
        В~треугольнике $ABC$ проведены чевианы $A A_1$, $B B_1$ и~$C C_1$,
        пересекающиеся в~точке~$P$.}
    \label{triangle-area-3:fig:generic-triangle}
\end{center}
\end{figure}

\emph{Вернемся к лемме о площадях!}

\begin{problems}

\item
\claim{Теорема ван Обеля}
Докажите, что $AP : P A_1 = A C_1 : C_1 B + A B_1 : B_1 C$.

\item
\claim{Теорема Жергонна}
Докажите, что $P A_1 : A A_1 + P B_1 : B B_1 + P C_1 : C C_1 = 1$.

\itemx{*}
Можно~ли сформулировать обратные теоремы?
(И будут~ли они верны? :)

\itemx{*}
Объясните, почему теоремы ван Обеля и~Жергонна~--- это на~самом деле одно
и~то~же геометрическое утверждение.

\end{problems}

