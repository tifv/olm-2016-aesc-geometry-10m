% $date: 2015-09-18

\section*{Теоремы Чевы и Менелая}

% $build$style[-print]:
% - .[colour-links]

\begin{figure}[hb]
\begin{center}
    \jeolmfigure[width=0.3\linewidth]{generic-triangle}
    \caption{%
        \emph{стандартные обозначения} в~этом листочке.
        В~треугольнике $ABC$ проведены чевианы $A A_1$, $B B_1$ и~$C C_1$,
        \emph{возможно,} пересекающиеся в~точке~$P$.}
    \label{triangle-area-2:fig:generic-triangle}
\end{center}
\end{figure}

\subsection*{Лемма о площадях (повторение)}

\begin{problems}

\item
Найдите отношение закрашенной площади к~площади всего треугольника:
\begin{center}
    \jeolmfigure[width=0.2\linewidth]{another-problem}
\end{center}

\item
В~треугольнике в~стандартных обозначениях чевианы пересекаются в~точке~$P$,
и~даны отношения $A B_1 : B_1 C = 2 : 1$, $C A_1 : A_1 B = 2 : 1$.
\\
\subproblem
Пусть площадь треугольника $ABC$ равна $7$.
Найдите площадь многоугольников $A B_1 P$, $A B_1 P C_1$.
\\
\subproblem
\ldots найдите площадь треугольника $A B_1 C_1$.
\\
\subproblem
Найдите, в~каких отношениях делят друг друга отрезки $AP$ и~$B_1 C_1$.

\end{problems}


\subsection*{Теорема Чевы}

\claim{Теорема Чевы}
Пусть в~треугольнике со~стандартными обозначениями чевианы пересекаются
в~одной точке.
Тогда верно равенство:
\[
    \frac{A C_1}{C_1 B} \cdot \frac{B A_1}{A_1 C} \cdot \frac{C B_1}{B_1 A}
=
    1
\, . \]

\claim{Обратная теорема Чевы}
Наоборот, если это равенство верно, то~проведенные чевианы
пересекаются в~одной точке.

\claim{Теорема Чевы в синусах}
Пусть в~треугольнике со~стандартными обозначениями чевианы пересекаются
в~одной точке.
Тогда верно равенство:
\[
    \frac{\sin \angle A C C_1}{\sin \angle C_1 C B} \cdot
    \frac{\sin \angle B A A_1}{\sin \angle A_1 A C} \cdot
    \frac{\sin \angle C B B_1}{\sin \angle B_1 B A}
=
    1
\, . \]

\emph{Обратная теорема Чевы в~синусах,} конечно, тоже верна.

\begin{problems}

\item
Докажите с~помощью теорем Чевы, что в~треугольнике
\\
\subproblem медианы;
\quad
\subproblem биссектрисы;
\quad
\subproblem высоты
\\
пересекаются в~одной точке.

\item
\subproblem
Пусть в~стандартных обозначениях
%$A B_1 = A C_1$, $B A_1 = B C_1$, $C A_1 = C B_1$.
точки $A_1$, $B_1$, $C_1$~--- это точки касания окружности, вписанной
в~треугольник $ABC$, со~сторонами треугольника.
Докажите, что чевианы пересекаются в~одной точке.
\\
\subproblem
Докажите, что в~произвольном треугольнике прямые, проходящие через вершины
и~делящие \emph{периметр} треугольника пополам, пересекаются в~одной точке.

\item
Пусть в~стандартных обозначениях чевианы пересекаются в~точке~$P$,
а~$C C_1$~--- медиана.
Докажите, что прямые $A_1 B_1$ и~$AB$ параллельны.

\end{problems}


\subsection*{Теорема Менелая}

\claim{Теорема Менелая}
Пусть прямая $\ell$ пересекает стороны $AB$, $BC$ и~$AC$ или их продолжения
в~точках $C_1$, $A_1$, $B_1$
соответственно.
Тогда верно равенство:
\[
    \frac{A C_1}{C_1 B} \cdot \frac{B A_1}{A_1 C} \cdot \frac{C B_1}{B_1 A}
=
    1
\, . \]

\claim{Обратная теорема Менелая}
Наоборот, если это равенство верно, то~точки $A_1$, $B_1$ и~$C_1$ лежат
на~одной прямой.

\begin{figure}[h]
\begin{center}
    \jeolmfigure[width=0.3\linewidth]{menelaus-triangle}
    \caption{к теореме Менелая}
    \label{triangle-area-2:fig:menelaus-triangle}
\end{center}
\end{figure}

\begin{problems}

\item
Докажите теорему Менелая, используя леммы о~площадях.
\\
\emph{Подсказка: на~картинке нет треугольника с~пересекающимися чевианами,
чтобы развернуться леммам о~площадях.
Но~если провести ещё пару линий\ldots}

\item
Используя прямую теорему Менелая, докажите обратную.

\end{problems}

