% $date: 2015-09-11

\section*{Две важные леммы}

% $build$style[-print]:
% - .[colour-links]

% $matter[-preamble-package-guard]:
% - preamble package: subcaption
% - .[preamble-package-guard]

\claim{Лемма об~отношении площадей треугольников с~общим углом}
\[
    \frac{S(A B C)}{S(A B' C')}
=
    \frac{A B}{A B'} \cdot \frac{A C}{A C'}
,\]
где точки $B$ и~$B'$ лежат на~одной стороне данного угла с~вершиной~$A$,
а~точки $C$ и~$C'$~--- на~другой стороне.

\begin{problems}

\item
Докажите лемму.

\end{problems}

\claim{Лемма об~отношении площадей треугольников с~общей стороной}
\[
    \frac{S(A B C)}{S(A B C')} = \frac{P C}{P C'}
,\]
где $P$~--- точка пересечения прямых $AB$ и~$CC'$.

\begin{problems}

\item
Докажите лемму.
Как она видоизменяется, если точка~$P$ не~существует (прямые параллельны)?
Не~поленитесь и~перечислите все частные случаи расположения точек $C$ и~$C_1$.

\end{problems}

\emph{%
\textbf{Внимание!}
В~следующих задачах можно пользоваться \emph{только} леммами о~площадях.}

\begin{figure}[hb]
\begin{center}
    \jeolmfigure[width=0.3\linewidth]{generic-triangle}
    \caption{%
        \emph{стандартные обозначения} в~этом листочке.
        В~треугольнике $ABC$ проведены чевианы $A A_1$, $B B_1$ и~$C C_1$,
        пересекающиеся в~точке~$P$.}
    \label{triangle-area-1:fig:generic-triangle}
\end{center}
\end{figure}

\begin{problems}

\item
В~треугольнике $ABC$ площади~$10$ проведены чевианы в~обозначениях
рис.~\ref{triangle-area-1:fig:generic-triangle}.
Известно, что $A B_1 : B_1 C = 1 : 2$, $B_1 P : P B = 2 : 3$.
Найдите
\\
\subproblem
площади треугольников $A B_1 B$, $ABP$, $BPC$, $APC$;
\\
\subproblem
отношения отрезков $B A_1 : A_1 C$, $A C_1 : C_1 B$, где $C_1$~--- точка
пересечения $AB$ и~$CP$;
\\
\subproblem
отношения отрезков $AP : P A_1$ и~$CP : P C_1$.

\item
В~стандартных обозначениях точек известны отношения
$C B_1 : B_1 A = 2 : 3$, $B A_1 : A_1 C = 2 : 1$.
Площадь треугольника $P C B_1$ равна $6$.
Найдите
\\
\subproblem
площади $A P B_1$, $ABP$, $BCP$;
\\
\subproblem
отношения отрезков $A C_1 : C_1 B$, $AP : P A_1$, $BP : P B_1$, $CP : P C_1$.

\item
В~стандартных обозначениях даны отношения $A B_1 : B_1 C = 2 : 3$
и~$AP : P A_1 = 5 : 2$.
Найдите все остальные отношения на~отрезках.
\emph{(Возьмите $S_{BCP} = 6$.)}

\item
В~стандартных обозначениях даны отношения $CP : P C_1 = 5 : 3$
и~$AP : P A_1 = 5 : 2$.
Найдите все остальные отношения на~отрезках.

\begin{figure}[hb]
\leavevmode\null\hfill
    \begin{subfigure}{0.3\textwidth}
        \jeolmfigure[height=0.8\linewidth]{problem-556}
        \caption{к~задаче~\ref{triangle-area-1:problem:556}.}
        \label{triangle-area-1:problem:556:fig}
    \end{subfigure}
\hfill
    \begin{subfigure}{0.3\textwidth}
        \jeolmfigure[height=0.8\linewidth]{problem-559}
        \caption{к~задаче~\ref{triangle-area-1:problem:559}.}
        \label{triangle-area-1:problem:559:fig}
    \end{subfigure}
\hfill
    \begin{subfigure}{0.3\textwidth}
        \jeolmfigure[height=0.8\linewidth]{problem-564}
        \caption{к~задаче~\ref{triangle-area-1:problem:564}.}
        \label{triangle-area-1:problem:564:fig}
    \end{subfigure}
\hfill\null\par
    \caption{}
\end{figure}

\item
\label{triangle-area-1:problem:556}%
Рис.~\ref{triangle-area-1:problem:556:fig}.
Найдите отношение $S_{C M_1 M M_2} : S_{ABC}$.

\item
\label{triangle-area-1:problem:559}%
Стороны треугольника $ABC$ площади $S$ удвоены, как показано
на~рис.~\ref{triangle-area-1:problem:559:fig}.
Найдите отношение $S_{A_1 B_1 C_1} : S$.

\item
\label{triangle-area-1:problem:564}%
Дан треугольник $ABC$ площади $S$.
На~сторонах $AB$, $BC$ и~$CA$ взяты точки $C_1$, $A_1$ и~$B_1$ соответственно
так, что
$A C_1 : C_1 B = 1 : 2$, $B A_1 : A_1 C = 2 : 3$, $C B_1 : B_1 A = 1 : 1$
(рис.~\ref{triangle-area-1:problem:564:fig}).
Найдите отношение $S_{A_1 B_1 C_1} : S$.

\end{problems}

