% $date: 2015-12-04

\section*{Вписанные углы --- 2}

% $authors:
% - Алексей Пономарев

\begin{problems}

\item
\subproblem
Из~точки~$A$, лежащей вне окружности, выходят лучи $AB$ и~$AC$, пересекающие
эту окружность.
Докажите, что величина угла $BAC$ равна полуразности угловых величин дуг
окружности, заключенных внутри этого угла.
\\
\subproblem
Хорды одной окружности $AB$ и~$CD$ пересекаются в~точке~$K$.
Докажите, что $\angle AKC$ равен полусумме угловых величин дуг $AC$ и~$BD$
(дуг, не~содержащих других отмеченных точек).

\item
Внутри треугольника $ABC$ взята точка~$P$ так, что
$\angle BPC = \angle A + 60^{\circ}$,
$\angle APC = \angle B + 60^{\circ}$,
$\angle APB = \angle C + 60^{\circ}$.
Прямые $AP$, $BP$ и~$CP$ пересекают описанную окружность треугольника $ABC$
в~точках $A_1$, $B_1$, $C_1$ соответственно.
Докажите, что треугольник $A_1 B_1 C_1$~--- правильный.

\item
Точки $A$, $B$, $M$, $N$ лежат на~окружности в~указанном порядке.
$A_1$, $B_1$~--- такие точки на~окружности, что
$NA \perp M B_1$, $NB \perp M A_1$.
Докажите, что $A A_1 \parallel B B_1$.

\item
В~неравнобедренном треугольнике $ABC$ проведена биссектриса~$AD$.
$E$~--- точка пересечения прямой~$BC$ с~касательной к~описанной окружности
исходного треугольника, восстановленной в~точке~$A$.
Докажите, что $AE = DE$.

\item
В~треугольнике $ABC$ проведена биссектриса~$AL$.
$K$~--- такая точка на~отрезке~$AC$, что $\angle BAC = \angle CLK$.
Докажите, что $BL = LK$.

\item
Две окружности пересекаются в~точках $P$ и~$Q$.
Прямая, проходящая через точку~$P$, второй раз пересекает первую окружность
в~точке~$A$, а~вторую~--- в~точке~$D$.
Прямая, проходящая через точку~$Q$ параллельно $AD$, второй раз пересекает
первую окружность в~точке~$B$, а~вторую~--- в~точке~$C$.
\\
\subproblem
Докажите, что четырехугольник $ABCD$~---параллелограмм.
\\
\subproblem
Найдите отношение $BP : PC$, если радиус первой окружности вдвое больше радиуса
второй.
\\
\emph{(МИОО, 2014)}

\item \emph{Лемма о~трезубце.}
\\
\subproblem
Дан треугольник $ABC$, точка~$I$~--- центр его вписанной окружности,
а~биссектриса угла~$A$ пересекает описанную окружность треугольника
в~точке~$L$.
Докажите, что $IL = BL = CL$.
\\
\subproblem
Пусть $I_A$~--- центр вневписанной окружности треугольника $ABC$
напротив вершины~$A$.
Докажите, что $IL = I_A L$.

\end{problems}

