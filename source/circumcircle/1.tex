% $date: 2015-11-20

\section*{Вписанные углы --- 1}

\subsection*{Ортоцентр}

В~остроугольном треугольнике $ABC$ проведены высоты $A A_1$, $B B_1$, $C C_1$,
пересекающиеся в~точке~$H$.

\begin{problems}

\item
\subproblem
Докажите, что точки $B$, $B_1$, $C_1$, $C$ лежат на~одной окружности.
\\
\subproblem
Докажите, что $\angle B_1 A_1 A = \angle A A_1 C_1$.

\item
Из~точки $A_1$ опущены высоты $A_1 C_2$ и~$A_1 B_2$ на~стороны $AB$ и~$AC$
соответственно.
\\
\subproblem
Докажите, что $A_1$, $A$, $B_2$, $C_2$ лежат на~одной окружности.
\\
\subproblem
Докажите, что точки $B$, $C$, $B_2$, $C_2$ лежат на~одной окружности.

\item
\subproblem
Точку~$H$ отразили симметрично относительно стороны $BC$ и~получили
точку~$H_{A}$.
Докажите, что $H_{A}$ лежит на~описанной окружности треугольника $ABC$.
\\
\subproblem
Точку~$H$ отразили относительно центрально-симметрично относительно середины
$A_1$ стороны $BC$ и~получили точку~$H_{A}'$
Докажите, что ее образ при отражении лежит на~описанной окружности
треугольника $ABC$.
\\
\subproblem
Докажите, что $H_{A} H_{A}' \parallel BC$.
\\
\subproblem
Докажите, что $A H_{A}'$~--- это диаметр описанной окружности $ABC$.

\end{problems}

\subsection*{Разные задачи}

\begin{problems}

\item
Высоты $B B_1$ и~$C C_1$ остроугольного треугольника $ABC$ пересекаются
в~точке~$H$.
\\
\subproblem
Докажите, что $\angle A H B_1 = \angle A C B$.
\\
\subproblem
Найдите $BC$, если $AH = 4$ и~$\angle BAC = 60^{\circ}$.
\\
\emph{(ЕГЭ, 2014)}

\item
На~диагонали параллелограмма взяли точку, отличную от~ее середины.
Из~нее на~все стороны параллелограмма (или их продолжения) опустили
перпендикуляры.
\\
\subproblem
Докажите, что четырехугольник, образованный основаниями этих перпендикуляров,
является трапецией.
\\
\subproblem
Найдите площадь полученной трапеции, если площадь параллелограмма равна $16$,
а~один из~его углов равен $60^{\circ}$.
\\
\emph{(МИОО, 2014)}

\end{problems}

