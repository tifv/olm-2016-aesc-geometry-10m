% $date: 2015-12-11

\section*{Вписанные углы --- 3}

% $authors:
% - Алексей Пономарев

\begin{problems}

\item
Окружности $S_1$ и~$S_2$ пересекаются в~точке~$A$.
Через точку~$A$ проведена прямая, пересекающая $S_1$ в~точке~$B$, а~$S_2$
в~точке~$C$.
В~точках $C$ и~$B$ проведены касательные к~окружностям, пересекающиеся
в~точке~$D$.
Докажите, что угол $\angle BDC$ не~зависит от~выбора прямой, проходящей
через $A$.

\item
Внутри квадрата $ABCD$ отметили точку~$E$ так, что
$\angle ECD = \angle EAC = \alpha$.
Найдите $\angle ABE$.

\item
$ABCD$~--- вписанный четырехугольник.
Лучи $AB$, $DC$ пересекаются в~точке~$P$, лучи $BC$, $AD$~--- в~точке~$Q$.
Докажите, что биссектрисы углов $APD$ и~$AQB$ перпендикулярны.

\item
$ABCD$~--- вписанный четырехугольник, $K$~--- середина дуги~$AB$, не~содержащей
точек $C$ и~$D$.
$P$ и~$Q$~--- точки пересечения пар хорд $CK$ и~$AB$, $DK$ и~$AB$
соответственно.
Докажите, что четырехугольник $CPQD$~--- вписанный.

\item
$O$~--- центр описанной окружности равнобокой трапеции $ABCD$ с~боковой
стороной $AB$, а~$K$~--- точка пересечения ее диагоналей.
Докажите, что точки $A$, $B$, $K$, $O$ лежат на~одной окружности.

\item
Точки $A$, $B$, $C$, $D$ лежат на~одной окружности в~указанном порядке.
$K$, $L$, $M$, $N$~--- середины дуг $AB$, $BC$, $CD$, $DA$, не~содержащих
внутри четырех исходных точек.
Докажите, что $KM \perp LN$.

\item
В~неравнобедренном треугольнике $ABC$ проведена биссектриса~$AD$.
$E$~--- точка пересечения прямой~$BC$ с~касательной к~описанной окружности
исходного треугольника, восстановленной в~точке~$A$.
Докажите, что $AE = DE$.

\end{problems}

