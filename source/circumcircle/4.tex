% $date: 2016-01-15

\section*{Вписанные углы --- 4}

% $authors:
% - Алексей Пономарев

\begin{problems}

\item
В~остроугольном треугольнике $ABC$ высоты $A A_1$, $B B_1$, $C C_1$
пересекаются в~точке~$H$.
Докажите, что $H$~--- точка пересечения биссектрис треугольника $A_1 B_1 C_1$.

\item
Основание~$A_1$ высоты~$A A_1$ треугольника $ABC$ отразили симметрично
относительно прямой~$AC$.
Докажите, что отраженная точка лежит на~прямой~$B_1 C_1$, где $B_1$ и~$C_1$~---
основания двух других высот треугольника.

\item
Точка~$O$~--- центр описанной окружности остроугольного треугольника $ABC$.
Обозначим $X$ основание перпендикуляра, опущенного из~$B$ на~$AO$.
Далее: $M$~--- середина $BC$, $A A_1$~--- высота треугольника $ABC$.
Докажите, что треугольник $X M A_1$~--- равнобедренный.

\item\emph{Теорема Симсона.}
На~описанной окружности треугольника $ABC$ отметили точку~$P$.
Докажите, что основания перпендикуляров, опущенных из~точки~$P$ на~стороны
треугольника, лежат на~одной прямой.

\item
В~остроугольном треугольнике $ABC$ проведены высоты $A A_1$, $B B_1$, $C C_1$.
Прямая, перпендикулярная стороне~$AC$ и~проходящая через точку~$A_1$,
пересекает прямую~$B_1 C_1$ в~точке~$D$.
Докажите, что угол $ADC$ прямой.

\item
В~остроугольном треугольнике $ABC$ на~продолжениях высот $BB_1$ и~$CC_1$
за~точки $B_1$ и~$C_1$ отметили точки $B_2$ и~$C_2$ соответственно.
Оказалось, что $\angle B_2AC_2 = 90^{\circ}$.
Из~точки $A$ опустили перпендикуляр $AX$ на~$B_2C_2$.
Докажите, что $\angle BXC = 90^{\circ}$.

\item
Докажите, что в~остроугольном треугольнике середины двух высот, основание
третьей и~ортоцентр лежат на~одной окружности.

\end{problems}

