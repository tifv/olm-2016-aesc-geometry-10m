% $date: 2015-10-30

\section*{Вневписанные окружности --- 3}

% $authors:
% - Алексей Пономарев

\begin{problems}

\item
Отрезок~$CD$ разбивает треугольник $ABC$ на~два треугольника, в~каждый
из~которых вписана окружность.
Пусть $I_1$ и~$I_2$~--- центры этих окружностей.
$E$ и~$F$~--- точки касания окружностей с~отрезком~$CD$.
\\
\subproblem
Докажите, что треугольник $I_1 D I_2$~--- прямоугольный.
\\
\subproblem
Выразите длину отрезка~$EF$ через $AC = b$, $CB = a$, $AD = x$, $BD = y$.
\\
Что получается, если $CD$~--- медиана? биссектриса?
\\
\subproblem
При каком расположении точки~$D$ окружности касаются друг друга?

\item
Картинка та~же, добавляем окружность, вписанную в~треугольник $ABC$.
Пусть $I$~--- ее центр.
$L$, $M$ и~$K$~--- точки касания со~стороной~$AB$ окружностей, вписанных
в~треугольники $ABC$, $ACD$ и~$BCD$ соответственно.
\\
\subproblem
Докажите, что $ML = DK$.
\\
\subproblem
Докажите, что окружности с~центрами $I_1$ и~$I_2$ касаются тогда и~только
тогда, когда $D$ и~$L$ совпадают.
%\\
%\subproblem\emph{(Шарыгин)}
%Докажите, что точки $I_1$, $L$, $D$ и~$I_2$ лежат на~одной окружности.

\item\emph{(Кожевников)}
Фиксированы две окружности, одна их внешняя касательная~$n$ и~одна внутренняя
касательная~$m$.
На~прямой~$m$ выбираем произвольную точку~$X$, а~на~прямой~$n$~---
точки $Y$ и~$Z$ так, что $XY$ и~$XZ$ касаются наших окружностей,
а~треугольник $XYZ$ их содержит.
Докажите, что центры окружностей, вписанных во~всевозможные такие
треугольники $XYZ$, лежат на~одной прямой.

\item\emph{(ЕГЭ)}
В~треугольнике $ABC$ заданы стороны $AB = 7$, $BC = 9$, $AC = 4$.
Точка~$D$ лежит на~прямой~$BC$ так, что $BD : DC = 1 : 5$.
Окружности, вписанные в~каждый из~треугольников $ADC$ и~$ADB$, касаются $AD$
в~точках $E$ и~$F$.
Найдите длину отрезка~$EF$.

\item
В~прямоугольном треугольнике с~катетами $5$ и~$12$ проведена медиана
к~гипотенузе, и~в~образовавшиеся треугольники вписаны окружности.
Найдите расстояние между точками касания этих окружностей с~гипотенузой.

\item
В~треугольнике со~сторонами $3$ и~$5$ и~углом $120^\circ$ между ними проведена
биссектриса к~третьей стороне, и~в~образовавшиеся треугольники вписаны
окружности.
Найдите расстояние между точками касания этих окружностей с~биссектрисой.

\end{problems}

