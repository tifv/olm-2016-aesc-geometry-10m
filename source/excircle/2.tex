% $date: 2015-10-23

\section*{Вневписанные окружности --- 2}

% $authors:
% - Алексей Пономарев

\begin{problems}

\item
Дан треугольник $ABC$.
Вписанная окружность касается стороны~$AC$ в~точке~$F$, а~вневписанная~---
в~точке~$E$.
\\
\subproblem
Выразите длину отрезка~$FC$ через стороны треугольника $ABC$.
\\
\subproblem
Докажите, что отрезки, соединяющие вершины треугольника $ABC$ и~точки касания
вписанной окружности с~противоположными сторонами, пересекаются в~одной точке
\emph{(точка Жергонна).}
\\
\subproblem
Выразите длину отрезка~$AE$ через стороны треугольника $ABC$.
\\
\subproblem
Докажите, что отрезки, соединяющие вершины треугольника $ABC$ и~точки касания
вневписанных окружностей с~противоположными сторонами, пересекаются в~одной
точке \emph{(точка Нагеля).}
\\
\subproblem
Докажите, что $AE = FC$.
\\
\subproblem
Найдите расстояние между точками касания вписанной и~вневписанной окружностей
со~стороной~$BC$.

\item
Постройте треугольник по~величине одного угла, длине высоты, проведенной
из~вершины этого угла, и~периметру.

\item
Пусть $r$~--- радиус вписанной окружности треугольника $ABC$,
а~$r_a$, $r_b$ и~$r_c$~--- радиусы его вневписанных окружностей.
\\
\subproblem
Докажите, что $r / r_{b} = (p - b) / p$, где $p$~--- полупериметр
треугольника $ABC$.
\\
\subproblem
Докажите, что
\(
    S_{ABC}
=
    p \cdot r
=
    (p - a) \cdot r_{a}
=
    (p - b) \cdot r_{b}
=
    (p - c) \cdot r_{c}
\)\,.
\\
\subproblem
Докажите, что $S_{ABC}^2 = r \cdot r_{a} \cdot r_{b} \cdot r_{c}$\,.
\\
\subproblem
Докажите, что $1 / r = 1 / r_{a} + 1 / r_{b} + 1 / r_{c}$\,.

\item
Докажите, что треугольник является прямоугольным с~прямым углом~$B$ тогда
и~только тогда, когда
\\
\subproblem $r_{b} = p$\,;
\qquad
\subproblem $S_{ABC} = r \cdot r_{b}$\,;
\qquad
\subproblem $S_{ABC} = r_{a} \cdot r_{c}$\,.

\end{problems}

