% $date: 2015-10-16

\section*{Вневписанные окружности --- 1}

% $authors:
% - Алексей Пономарев

\begin{problems}

\item
Найдите геометрическое место точек, равноудаленных от~сторон
треугольника $ABC$.

\item
Дан треугольник $ABC$.
Точка~$I_{B}$~--- центр вневписанной окружности, касающейся стороны~$AC$.
Найдите угол $A I_{B} C$, если угол $ABC$ равен $\beta$.

\item
Постройте треугольник $ABC$, зная положение трех точек $A_1$, $B_1$ и~$C_1$~---
центров его вневписанных окружностей.

\item
В~треугольнике $ABC$ проведены биссектрисы $AD$ и~$BE$.
Оказалось, что $DE$~--- биссектриса треугольника $ADC$.
Найдите угол $BAC$.

\item\emph{(треугольник Шебаршина)}
В~треугольнике $ABC$ с~углом~$B$, равным $120^{\circ}$\!, проведены биссектрисы
$A A_1$, $B B_1$ и~$C C_1$.
Докажите, что треугольник $A_1 B_1 C_1$~--- прямоугольный.

\item\emph{(Волчкевич)}
Углы, прилежащие к~одной из~сторон треугольника, равны $15^{\circ}$
и~$30^{\circ}$\!.
Какой угол образует с~этой стороной проведенная к~ней медиана?

\item
Точка $E$ на~стороне~$AD$ квадрата $ABCD$ такова, что угол $AEB$ равен
$60^{\circ}$.
Биссектриса угла $ABE$, отразившись от~стороны~$AD$, пересекает отрезок~$BE$
в~точке~$F$.
Докажите, что точка~$F$ лежит на~диагонали квадрата.

\item\emph{(Произволов)}
На~полосу наложили квадрат, сторона которого равна ширине полосы, так, что его
граница пересекает границы полосы в~четырех точках.
Докажите, что две прямые, проходящие крест-накрест через эти точки,
пересекаются под углом $45^{\circ}$.

%\item
%На~стороне~$BC$ квадрата $ABCD$ выбрана точка~$M$, а~на~стороне~$CD$~---
%точка~$K$ так, что угол $MAK$ равен $45^{\circ}$.
%Докажите, что расстояние от~вершины~$A$ до~прямой~$MK$ равно стороне квадрата.

\end{problems}

