% $date: 2016-04-29

\section*{Плоскости и прямые --- 2}

\subsection*{Recap}

\emph{Угол между прямыми} не~меняется при параллельном переносе прямых.
Чтобы вычислить угол между скрещивающимися прямыми, их нужно сдвинуть так,
чтобы они пересеклись.

\emph{Прямая и~плоскость перпендикулярны,} если прямая перпендикулярна каждой
прямой, лежащей в~плоскости.

При этом, если прямая перпендикулярна каким-то двум пересекающимся прямым,
лежащим в~плоскости, то~она перпендикулярна этой плоскости.

\subsection*{Задачи}

\begin{problems}

\item
Дан куб $A B C D A_1 B_1 C_1 D_1$.
Пусть $M$~--- центр квадрата $ABCD$.
Найдите углы между прямыми:
\\
\subproblem
$A_1 M$ и~$B_1 D_1$;
\qquad
\subproblem
$A_1 M$ и~$B_1 C_1$;
\qquad
\subproblem
$A_1 M$ и~$A C_1$.

\item
Дан правильный тетраэдр $ABCD$ с~длиной ребра $1$.
Пусть $M$~--- середина ребра~$CD$.
\\
\subproblem
Докажите, что $AM \perp CD$.
\\
\subproblem
Докажите, что $AB \perp CD$.
\\
\subproblem
Найдите угол между прямыми $AM$ и~$AC$.
\\
\subproblem
Найдите угол между прямыми $AM$ и~$AB$.
\\
\subproblem
Пусть точка $T$ такова, что $BATC$~--- параллелограмм.
Чему равна длина отрезка~$TD$?
\\
\subproblem
Найдите угол между прямыми $AM$ и~$BC$.

\item \emph{Теорема о~трёх перпендикулярах.}
Дана плоскость~$\alpha$, прямая~$b$ в~этой плоскости, и~точка~$C$ вне этой
плоскости.
Точка~$D$~--- проекция точки~$C$ на~плоскость $\alpha$, то~есть $D \in \alpha$
и~$CD \perp \alpha$.
Пусть $B \in b$.
Докажите, что $DB \perp b$ тогда и~только тогда, когда $CB \perp b$.

\end{problems}

