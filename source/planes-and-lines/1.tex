% $date: 2016-04-15

\section*{Плоскости и прямые --- 1}

\begin{problems}

\item
Докажите, что если прямые $AB$ и $CD$ скрещиваются, то прямые $AD$ и $BC$ тоже
скрещиваются.

\item
Точки $A$, $B$, $C$, $D$ не лежат в одной плоскости \emph{(некомпланарны)}.
\\
\subproblem
Могут ли три из них лежать на одной прямой?
\\
\subproblem
Докажите, что середины отрезков $AB$, $BC$, $CD$, $DA$ образуют
параллелограмм.

\end{problems}

\subsection*{Теорема Фалеса и аналоги}

\emph{В каждой плоскости работает планиметрия.}

\begin{problems}

\item
Даны две плоскости $a$ и $b$, пересекаемые тремя
прямыми $\ell_1$, $\ell_2$, $\ell_3$ соответственно
в точках $A_1$, $A_2$, $A_3$ и $B_1$, $B_2$, $B_3$.
Прямые $\ell_1$, $\ell_2$, $\ell_3$ лежат в одной плоскости.
Докажите, что $A_1 A_2 : A_2 A_3 = B_1 B_2 : B_2 B_3$.

\item
Даны две прямые $a$ и $b$, пересекаемые тремя
плоскостями $\pi_1$, $\pi_2$, $\pi_3$ соответственно
в~точках $A_1$, $A_2$, $A_3$ и $B_1$, $B_2$, $B_3$.
Плоскости $\pi_1$, $\pi_2$, $\pi_3$ попарно параллельны.
Докажите, что $A_1 A_2 : A_2 A_3 = B_1 B_2 : B_2 B_3$, если
\\
\subproblem
прямые $a$ и $b$ лежат в одной плоскости;
\\
\subproblem
прямые $a$ и $b$ скрещиваются.

\item
Даны две параллельные прямые $a$ и $b$, пересекаемые тремя
плоскостями $\pi_1$, $\pi_2$, $\pi_3$ соответственно
в точках $A_1$, $A_2$, $A_3$ и $B_1$, $B_2$, $B_3$.
Плоскости $\pi_1$, $\pi_2$, $\pi_3$ пересекаются по~прямой~$\ell$.
Докажите, что $A_1 A_2 : A_2 A_3 = B_1 B_2 : B_2 B_3$.

\end{problems}

\subsection*{Треугольники и проекции}

\begin{problems}

\item
В пространстве на некоторой плоскости нарисовали треугольник $ABC$ и отметили
его точку пересечения медиан, $M$.
Кроме того, в пространстве отметили прямую~$\ell$ и плоскость~$\pi$,
пересекающиеся.
\\
\subproblem
Через точку~$A$ провели прямую, параллельную $\ell$.
Она пересекла плоскость~$\pi$ в точке~$A'$.
Аналогично построили точки $B'$ и $C'$, а $M'$ построить забыли.
После этого плоскость~$\pi$ (с отмеченными точками) отдали Васе.
Как ему определить, где на плоскости должна быть точка~$M'$?
\\
\subproblem
Через точку~$A$ провели плоскость, параллельную $\pi$.
Она пересекла прямую~$\ell$ в точке~$A_1$.
Аналогично построили точки $B_1$ и $C_1$, а $M_1$ построить забыли.
После этого прямую~$\ell$ (с отмеченными точками) отдали Маше.
Как ей определить, где на прямой должна быть точка~$M_1$?

\item
Три прямые, проходящие через одну точку и не лежащие в одной плоскости,
пересекают одну из параллельных плоскостей в точках $A_1$, $B_1$, $C_1$,
а другую~--- в точках $A_2$, $B_2$, $C_2$.
Докажите, что треугольники $A_1 B_1 C_1$ и $A_2 B_2 C_2$ подобны.

\end{problems}

