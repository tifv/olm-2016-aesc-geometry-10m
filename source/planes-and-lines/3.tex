% $date: 2016-05-13

\section*{Плоскости и прямые --- 3}

\subsection*{Recap}

\emph{Две плоскости перпендикулярны,} если одна из них содержит прямую,
перпендикулярную другой.

\emph{Угол между прямой и плоскостью}~--- это угол между прямой и ее проекцией
на плоскость.

\emph{Угол между плоскостями}~--- это угол, который образуется в их сечении,
перпендикулярном их общей прямой.

\subsection*{Задачи}

\begin{problems}

\item
Даны две плоскости $\alpha$ и $\beta$, пересекающиеся под углм $\psi$.
Какие значения может принимать угол между прямыми $a$ и $b$, если
$a \subset \alpha$ и $b \subset \beta$?

\item
Дан куб $A B C D A_1 B_1 C_1 D_1$.
В нем отмечены плоскости $\alpha = ABCD$, $\beta = A B C_1 D_1$
и $\gamma = A B_1 D_1$;
а также прямые $a = A A_1$, $b = B A_1$, $c = C A_1$.
\\
\subproblem
Найдите углы между $\alpha$ и $b$, $\alpha$ и $c$, $\alpha$ и $\gamma$.
\\
\subproblem
Докажите, что $\alpha \perp a$, $\beta \perp b$, $\gamma \perp c$.
\\
\subproblem
Докажите, что углы между плоскостями $\alpha$, $\beta$ и $\gamma$ равны
соответственно углам между прямыми $a$, $b$, $c$.
\\
\subproblem
Найдите углы между $\beta$ и $a$, $\beta$ и $c$, $\beta$ и $\gamma$.

\end{problems}

